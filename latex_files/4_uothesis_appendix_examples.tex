%------------------------------------------------------------------------------%
% APPENDICES
%------------------------------------------------------------------------------%
\appendix
% Changes the table and figure counting to A.1 style
\renewcommand\thefigure{\thechapter.\arabic{figure}}
\renewcommand\thetable{\thechapter.\arabic{table}}

%--- Appendix A ---------------------------------------------------------------%
\chapter{IRIS-HEP Fellowship Project: Analytical Tracking Validation}
\section{Overview of the Project}
\subsection{Motivation and Goals}
\subsection{A Common Tracking Software (ACTS) framework and Open Data Detector (ODD) Geometry}
\subsection{Python Tracking Resolution Calculator}
\section{Methodology}

\subsection{Analytical Resolution Formulation}
The analytical model computes the uncertainties of the five standard ACTS track
parameters: $(d_0, z_0, \phi, \theta, q/p)$. The covariance matrix is obtained
from closed-form expressions derived from linearized track fits in the presence
of Gaussian measurement errors and small-angle multiple scattering.

\subsection{Measurement and Multiple Scattering Terms}
The measurement term scales according to the intrinsic hit resolutions and the
lever arm of the detector. It typically dominates at high momenta where
multiple scattering is negligible.

The multiple scattering contribution follows the Highland approximation,
\[
\theta_{\mathrm{ms}} \simeq \frac{13.6~\mathrm{MeV}}{\beta p}
\sqrt{\frac{x}{X_0}}
\left[1 + 0.038 \ln\left(\frac{x}{X_0}\right)\right],
\]
where $x/X_0$ is the material thickness. This term becomes dominant at low
$p_T$ and is sensitive to the accuracy of the material map.

One recurring challenge observed during the project was interpreting the
material distribution in the ODD geometry. Differences between the analytical
material assumptions and the more detailed ACTS material description were found
to contribute significantly to discrepancies in $\sigma(d_0)$ and
$\sigma(z_0)$ at low $p_T$.

\subsection{Matrix Representation of Track Parameters}
Track resolutions are determined by inverting the normal-equation matrix
associated with the linearized track model. The calculator constructs both the
design matrix and the noise matrix using detector layer positions and
resolutions, then derives the covariance matrix via:
\[
\mathbf{C} = (\mathbf{A}^\top \mathbf{W} \mathbf{A})^{-1},
\]
where $\mathbf{W}$ contains both measurement and scattering weights. This
framework allows the analytical model to remain computationally light while still
capturing key geometric dependencies.

\section{ACTS Configuration and Simulation Setup}
The ACTS validation used the full ODD geometry with a $B=2$~T magnetic field and
default digitization and reconstruction settings. Tracks were generated at fixed
$p_T$ values ranging from 1~GeV to 200~GeV and at several representative
pseudorapidity values. The resulting reconstructed track parameters and
covariance matrices were extracted from \texttt{tracksummary.ckf.root} and
converted into flat tables for comparison.

This workflow mirrors typical detector performance studies in ATLAS and ensures
that the results incorporate realistic navigation, material interactions, and
Kalman filter behavior.

\section{Comparison Strategy}
For each $(p_T, \eta)$ point, the analytical prediction for each parameter's
resolution was compared with the RMS width (or Gaussian core width) of the ACTS
simulation. The comparison was performed for:
\begin{itemize}
    \item $\sigma(d_0)$
    \item $\sigma(z_0)$
    \item $\sigma(\phi)$
    \item $\sigma(\theta)$
    \item $\sigma(p_T)/p_T$
\end{itemize}

Residual fractional differences were computed to highlight systematic trends,
and discrepancies were traced back to underlying assumptions such as:  
material description, hit resolution mapping, and the treatment of scattering
correlations within the Kalman filter.

% -------------------------------------------------------

\section{Results and Discussion}

\subsection{Resolution vs. $p_T$ and $\eta$}
Across most $p_T$ values, the analytical model and ACTS simulation show good
agreement in the high-momentum regime, where measurement errors dominate. The
$p_T$ dependence follows the expected scaling $\sigma \propto 1/p_T$ for the
momentum resolution and constant behavior for angular uncertainties.

At central $\eta \approx 0$, the agreement in $\sigma(d_0)$ and $\sigma(z_0)$ is
excellent above approximately $20$~GeV. However, at low $p_T$, ACTS resolutions
rise more steeply than predicted, reflecting stronger real-material multiple
scattering effects.

\subsection{Discrepancies and Model Validation}
The largest discrepancies appear in regions with significant material
interactions, particularly the endcap regions and low-momentum regimes. These
differences were traced to:
\begin{itemize}
    \item Simplified analytical treatment of the ODD material map,
    \item Non-uniform detector layer spacing in the endcaps,
    \item Additional scattering terms handled automatically by the ACTS Kalman
          filter but absent in the analytical model,
    \item Slight differences in hit resolution interpretation between ACTS and
          the analytical tool.
\end{itemize}

Despite these differences, the analytical model captures the overall behavior
and correctly predicts the scale of the resolutions.

\subsection{Implications for Fast Performance Estimation}
The study confirms that analytical resolution models are highly effective for
quick performance estimates in the central detector region and in the
high-momentum regime. However, caution is required when applying these models in
low-$p_T$ or high-$\eta$ regions where realistic material and scattering effects
dominate.

This validation provides meaningful guidance for future detector optimization
workflows, and it highlights the importance of maintaining consistency between
analytical assumptions and full simulation frameworks such as ACTS.


% ================================================================
\section{Conclusions}
\subsection{Summary of Findings}
\subsection{Cross-Project Insights}
\subsection{Future Work}

% ================================================================

\section{Supplementary Material}
\subsection{ACTS and \texttt{Traccc} Configuration Files}
\subsection{Code Snippets and YAML Settings}
\subsection{Additional Figures}


% %--- Appendix B ---------------------------------------------------------------%
% % Restart counting to B.1, B.2...
% \setcounter{figure}{0}
% \setcounter{table}{0}
% \chapter{The Second Appendix}
% \section{Appendix Two Section One}
% \subsection{Chapter two section one sub-section one}

