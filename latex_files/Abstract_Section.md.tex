The first study focuses on the development and validation of a GPU-based strip clustering algorithm implemented within the \texttt{Traccc} framework. Designed for the high-luminosity environment of ATLAS Run 4 data, the algorithm reconstructs hit clusters from silicon strip sensors and is designed for integration with the ATLAS Event Filter (EF) tracking chain in the Athena framework. Because the same clustering implementation in \texttt{Traccc} can also be used for offline reconstruction, its interoperability is important for both trigger and offline workflows. Its GPU-oriented design aims to improve throughput for high-pileup conditions expected at the High-Luminosity LHC. By comparing the differences in local x and y coordinates between \texttt{Traccc} and Athena, the results show good consistency in strip clustering performance in \texttt{Traccc}.

The second study investigates the sensitivity of the dark photon search in the process $ggH \rightarrow \gamma\gamma_D$, using Monte Carlo data. The analysis aims to optimize the selection criteria on key variables to maximize the signal significance by studying their individual performance distributions, receiver operating characteristic (ROC) curves, and the impact of variable thresholds on overall significance. A Machine Learning (ML) classifier (XGBoost BDT) study was also investigated to further enhance the significance of the signal over the backgrounds.  

The third component of this thesis (in Appendix), conducted as part of the Institute for Research and Innovation in Software for High Energy Physics (IRIS-HEP) Fellowship, validates a fast analytical tracking resolution calculator against full ACTS reconstruction using the Open Data Detector geometry. Analytical predictions of resolution of the track parameters $\sigma(d_0)$, $\sigma(z_0)$, $\sigma(\theta)$, $\sigma(\phi)$ and $\sigma(p_T)/p_T$ were compared to ACTS simulations across a range of transverse momenta and pseudorapidities of the particle gun, revealing systematic differences attributable to multiple-scattering modeling and detector material assumptions.


