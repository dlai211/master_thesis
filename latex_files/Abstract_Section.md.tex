This thesis presents two independent studies: a physics analysis of the ATLAS Run 3 proton-proton collision data and a strip-clustering algorithm designed for the high-luminosity environment of the ATLAS experiment at the Large Hadron Collider (LHC). 

The first study focuses on the development and validation strip-detector clustering algorithm in \testtt{Traccc}, a GPU-accelerated tracking toolkit (usable for both offline track reconstruction and the online trigger system) designed to outperform traditional CPU-based algorithms. To support more collision data per second in the next LHC upgrade. To support more data processing in a higher-collision-per-second environment during the Phase-II upgrade program for the High-Luminosity LHC (HL-LHC), we proposed Event Filter (EF) Tracking GPU pipelines that use fully or partially GPU-based algorithms. The strip clusterization algorithm groups cell hits from silicon strip sensors to form measurements and is integrated into the Event Filter (EF) Tracking GPU pipeline in the Athena framework. The result demonstrates good consistency when comparing the differences in local measurement positions between \texttt{Traccc} and Athena. 

Alongside this detector-focused work, I collaborate with a postdoctoral researcher and a senior graduate student in our group to search for Higgs boson decays to photons + dark photons produced via gluon-gluon fusion, using simulated data based on an integrated luminosity of $135~\mathrm{fb}^{-1}$ of ATLAS Run3 data. Motivated by finding a potential candidate for dark matter, we search for events with the final state of a Standard Model (SM) photon and missing transverse energy ($E_{\mathrm{T}}^{\mathrm{miss}}$). My analysis aims to maximize the signal-to-background significance by optimizing selection criteria based on distributions of various variables. A Machine Learning (ML) classifier study was also conducted to further enhance signal sensitivity. After optimizing the selection criteria, the analysis reduces the expected upper limit on the branching ratio of this process from the previous study by $20\%$ (from $1.1$ to $0.88$). 




