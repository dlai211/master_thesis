%--- Chapter 1 ----------------------------------------------------------------%
\chapter{Introduction}
\section{The Standard Model of particle physics}
\subsection{Brief Overview of SM particles and interactions}

The Standard Model of particle physics is a relativistic quantum field theory that describes the elementary particles known to date that constitute matter and the four fundamental forces of nature (except gravity). At the Large Hadron Collider (LHC) in 2012, the latest particle discovery was the Higgs boson, which is responsible for electroweak symmetry breaking and mass generation. ~\cite{ATLAS:2023oaq}

This chapter will introduce the components and their significance in the Standard Model, while the next section, beyond the Standard Model, will explain the limitations of the current model. 

\begin{figure}[htb]
    \centering
    \includegraphics[width=0.8\textwidth]{figures/Standard_Model_of_Elementary_Particles.svg.png}
    \caption{Elementary particles of the Standard Model \cite{WikiSM2018}}
    \label{fig:example}
\end{figure}


The Standard Model is based on the quantum field theory described by the symmetry group of
\[
SU(3)_C \times SU(2)_L \times U(1)_Y.
\]


Particles in the Standard Model are categorized into fermions and bosons. Each particle has its own antiparticle with opposite charges, while all other properties remain the same. ~\cite{Griffiths_EPP} ~\cite{Thomson_MPP}

Fermions are building blocks of matter that have $\frac{1}{2}$ spin and consist of three generations of quarks and leptons. These particles obey Fermi-Dirac statistics and are subject to the Pauli exclusion principle, which states that no two fermions can occupy the same quantum state. It explains that electrons move to higher-energy states when lower-energy states are occupied, thereby forming the atomic structure we know today. 
The six flavors of quarks are up, down, charm, strange, top, and bottom quarks. Organized by generations, and the increasing mass from left to right represents 
\[
\begin{pmatrix} u \\ d \end{pmatrix},
\begin{pmatrix} c \\ s \end{pmatrix},
\begin{pmatrix} t \\ b \end{pmatrix}
\]
First row of the quarks possess a fractional charges of +$\frac{2}{3}$; second row has -$\frac{1}{3}$. Their corresponding antiparticles are
\[
\begin{pmatrix} \bar{u} \\ \bar{d} \end{pmatrix},
\begin{pmatrix} \bar{c} \\ \bar{s} \end{pmatrix},
\begin{pmatrix} \bar{t} \\ \bar{b} \end{pmatrix}
\]
All these quarks have a non-Abelian charge called color charge from the local gauge symmetry $SU(3)_c$ of quantum chromodynamics (QCD). Due to color confinement, they cannot exist by themselves, leaving them in the form of color-neutral particles, known as hadrons. Hadrons are further categorized into mesons (two quarks) and baryons (three quarks). Mesons consist of a quark-antiquark pair, like the triplets of pions ($\pi^+$ , $\pi^0$ , $\pi^-$), where the color cancels out a color from the quark and its corresponding anti-color from the antiquark. Examples of baryons are protons (p) and neutrons (n), where the color-neutral raised from the balance of the three colors of each quark, commonly denoted as red, green, and blue.  Quarks interact through all four fundamental forces. Strong force mediated by gluons binds quarks into hadrons; the electromagnetic and weak interactions are responsible for charge and flavor change; the gravitational force, due to its force magnitude, is irrelevant compared to the other three forces. 

The leptons are the electron, muon, tau, electron neutrino, muon neutrino, and tau neutrino. Organized by generations and increasing mass from left to right represents 
\[
\begin{pmatrix} e^- \\ \nu_{e} \end{pmatrix},
\begin{pmatrix} \mu^- \\ \nu_{\mu} \end{pmatrix},
\begin{pmatrix} \tau^- \\ \nu_{\tau} \end{pmatrix},
\]
The first row of leptons has a charge of -1, and neutrinos are charge-neutral and left-handed. Their corresponding anti-particles are 
\[
\begin{pmatrix} e^+ \\ \bar{\nu_{e}} \end{pmatrix},
\begin{pmatrix} \mu^+ \\ \bar{\nu_{\mu}} \end{pmatrix},
\begin{pmatrix} \tau^+ \\ \bar{\nu_{\tau}} \end{pmatrix},
\]
The first row of anti-leptons becomes +1 charge, and anti-neutrinos are right-handed.

Charged leptons interact via the electromagnetic, weak, and gravitational interactions, whereas neutrinos interact only via the weak interaction and gravity, making them weakly interacting and difficult to detect. Lepton flavor is conserved in electromagnetic and strong processes, but can change through weak interactions, a phenomenon most strikingly observed in neutrino oscillations, which demonstrate that neutrinos have nonzero but tiny masses. The increasing masses across generations, particularly the large tau mass relative to the electron, reflect a hierarchical structure whose origin remains unexplained within the Standard Model.

Bosons comprise four spin-1 gauge bosons and one spin-0 Higgs Boson. The gauge bosons are fundamental force mediators: the photon for electromagnetism, the W and Z bosons for weak interaction, and the gluons for the strong interaction. A defining feature of bosons is Bose–Einstein condensation, where numerous identical bosons can all occupy the lowest-energy state. These particles transmit forces through quantum field interactions and are responsible for binding charged particles, inducing particle decays, and confining quarks inside hadrons. Notice that there is no gauge boson for gravitation. Even though theory proposes a massless spin-2 graviton as the force carrier, we have not yet experimentally proven it. In addition to gauge bosons, the Standard Model contains the Higgs boson, a scalar (spin-0) particle associated with the Higgs field, whose nonzero vacuum expectation value gives rise to the masses of elementary fermions and weak gauge bosons through spontaneous electroweak symmetry breaking. 


---

\subsection{Role of the Higgs boson}
The Higgs boson is a fundamental scalar particle associated with the Higgs field, which is introduced in the Standard Model to explain the origin of mass for elementary particles while preserving gauge invariance. ~\cite{Thomson_MPP} The Higgs field is a complex scalar doublet transforming under the electroweak gauge group \(SU(2)_L \times U(1)_Y\), and its dynamics are governed by the Higgs potential
\begin{equation}
V(\Phi) = \mu^2 \Phi^\dagger \Phi + \lambda (\Phi^\dagger \Phi)^2,
\end{equation}
where \(\mu^2 < 0\) and \(\lambda > 0\). The negative quadratic term leads to spontaneous symmetry breaking, causing the Higgs field to acquire a nonzero vacuum expectation value.

Minimizing the potential yields
\begin{equation}
\langle \Phi \rangle =
\frac{1}{\sqrt{2}}
\begin{pmatrix}
0 \\
v
\end{pmatrix},
\end{equation}
where \(v = \sqrt{-\mu^2 / \lambda} \approx 246~\text{GeV}\) is the electroweak vacuum expectation value. This vacuum configuration breaks the electroweak symmetry down to the electromagnetic subgroup \(U(1)_{\mathrm{EM}}\).

As a consequence of electroweak symmetry breaking, the weak gauge bosons acquire mass through their interactions with the Higgs field. The resulting masses are given by
\begin{equation}
m_W = \frac{1}{2} g v, \qquad
m_Z = \frac{1}{2} \sqrt{g^2 + g'^2}\, v,
\end{equation}
while the photon remains massless. Fermions acquire mass through Yukawa interactions of the form
\begin{equation}
\mathcal{L}_{\mathrm{Yukawa}} = - y_f \bar{\psi}_L \Phi \psi_R + \mathrm{h.c.},
\end{equation}
leading to fermion masses
\begin{equation}
m_f = \frac{y_f v}{\sqrt{2}},
\end{equation}
where \(y_f\) denotes the Yukawa coupling of fermion \(f\).

The Higgs boson itself arises as a physical excitation around the vacuum expectation value of the Higgs field. Its mass is determined by the curvature of the Higgs potential at the minimum,
\begin{equation}
m_H = \sqrt{2 \lambda}\, v,
\end{equation}
and has been experimentally measured to be approximately \(125~\text{GeV}\). Unlike gauge bosons, the Higgs boson does not mediate a force, but instead plays a central structural role by setting the mass scale of the Standard Model. Its couplings are proportional to particle masses, making it a unique probe of the electroweak symmetry-breaking mechanism and a sensitive window into possible physics beyond the Standard Model.

% \subsection{Experimental validation at the LHC}

---

\section{Beyond the Standard Model}
\subsection{Motivations for BSM searches}

The Standard Model of particle physics provides a highly successful description of the known elementary particles and their interactions, and its predictions have been confirmed by numerous precision measurements. Nevertheless, the Standard Model is not a complete theory of fundamental physics. Both theoretical considerations and experimental observations point to several unresolved problems that motivate the search for physics beyond the Standard Model. ~\cite{PDG2024} ~\cite{Thomson_MPP}

One of the most significant shortcomings of the Standard Model is its inability to incorporate gravity. While the electromagnetic, weak, and strong interactions are described within a quantum field theoretical framework, gravity is currently understood only through classical mechanics and general relativity. Even though theory predicts the existence of a force carrier for gravity, it has yet to be experimentally verified. Therefore, the absence of a consistent quantum theory of gravity highlights a fundamental limitation of the Standard Model and suggests that further discoveries are needed. 

The Standard Model also fails to account for the existence and properties of dark matter, which makes up around 27\% of the whole universe. Astrophysical and cosmological observations, including galaxy rotation curves, gravitational lensing, and measurements of the cosmic microwave background, provide compelling evidence for a non-luminous form of matter that constitutes a significant fraction of the energy density of the universe. However, none of the particles contained in the Standard Model possess the necessary properties to serve as a viable dark matter candidate.

Another unresolved issue is the origin of neutrino masses. In the Standard Model, neutrinos are massless due to the absence of right-handed neutrino fields and Yukawa couplings. Experimental observations of neutrino oscillations, however, demonstrate that neutrinos have nonzero masses and oscillate between flavor states. Explaining these masses requires introducing new particles, new interactions, or higher-dimensional operators beyond the Standard Model.

The pattern of fermion masses and mixing angles, commonly referred to as the flavor problem, also remains unexplained. All matter in the Standard Model is organized into three generations, but there is no theoretical reason why it has be exactly three. Secondly, the mass comes from the Higgs coupling. The coupling strength is a free parameter for which the value for the mass comes from nowhere. There is no pattern or formula derived from first principles that predicts these masses. Moreover, particles can change flavor via the weak interaction. Why would quarks stick to their own generation, while neutrinos oscillate between flavors constantly? The Standard Model does not explain why the mixing patterns are so different between the two types of fermions. 

The Higgs sector introduces further theoretical challenges. Radiative corrections to the Higgs boson mass are sensitive to energy scales far above the electroweak scale, leading to the hierarchy or naturalness problem. Stabilizing the Higgs mass without extreme fine-tuning suggests the existence of new physics near the electroweak scale or new mechanisms that protect scalar masses from large quantum corrections.

In addition, the Standard Model does not explain the observed matter--antimatter asymmetry of the universe. Although it contains sources of CP violation in the quark sector, these effects are insufficient to account for the baryon asymmetry generated in the early universe. Additional sources of CP violation or departures from thermal equilibrium are therefore required.

Finally, the Standard Model provides no explanation for the values of its fundamental parameters, including gauge couplings, particle masses, and mixing angles. These parameters must be determined experimentally, limiting the theory's predictive power and suggesting that it may be an effective description of a deeper underlying framework.

Taken together, these open questions provide strong motivation for the study of physics beyond the Standard Model. Addressing these issues requires extending the particle content, symmetries, or interactions of the theory, often leading to new degrees of freedom with weak or indirect couplings to Standard Model particles. Such extensions offer concrete experimental signatures that can be tested at collider experiments and in precision measurements, forming the basis for the investigations discussed in the following section.

---

\section{ATLAS Run 3 Detector Configuration}
\subsection{Introduction}

The A Toroidal LHC ApparatuS (ATLAS) is a general-purpose particle physics experiment located at Interaction Point 1 (IP1) of the Large Hadron Collider (LHC). Designed to exploit the full discovery potential of the LHC, ATLAS is capable of detecting a wide range of physics signatures, from high-precision Standard Model measurements to searches for new particles such as heavy bosons, microscopic black holes, or dark matter candidates.

The detector is forward-backward symmetric with respect to the interaction point. It measures approximately 46 meters in length, 25 meters in diameter, and weighs roughly 7,000 tonnes. ATLAS is composed of concentric layers of sub-detectors, each designed to measure specific properties of particles produced in proton-proton collisions.

---

\subsection{Timeline of LHC and ATLAS Data-Taking Runs}

The physics program of the ATLAS experiment is organized into successive operational periods of the Large Hadron Collider (LHC), known as runs, which are separated by long shutdowns dedicated to accelerator consolidation and detector upgrades. These runs---Run~1, Run~2, Run~3, and the coming Run~4---correspond to distinct experimental eras with progressively increasing collision energy, luminosity, and detector performance requirements.

---

\subsubsection{Run 1 (2010--2012): }

Run~1 marked the first physics operation of the LHC, delivering proton--proton collisions at center-of-mass energies of $\sqrt{s}=7~\mathrm{TeV}$ (2010--2011) and $8~\mathrm{TeV}$ (2012). During this period, the ATLAS detector was fully commissioned, and baseline performance for tracking, calorimetry, muon reconstruction, and triggering was established.

The most significant scientific achievement of Run~1 was the discovery of the Higgs boson in 2012 by the ATLAS and CMS collaborations. From an experimental perspective, Run~1 validated the overall detector design while highlighting limitations in radiation tolerance, trigger bandwidth, and tracking performance that informed later upgrades.

---

\subsubsection{Run 2 (2015--2018): }

Run~2 saw the LHC operate at a center-of-mass energy of $\sqrt{s}=13~\mathrm{TeV}$, nearly doubling the energy available in Run~1. This enabled precision measurements of Higgs boson properties, detailed studies of top-quark and electroweak processes, and extensive searches for physics beyond the Standard Model.

A major ATLAS upgrade for Run~2 was the installation of the Insertable B-Layer (IBL), which significantly improved vertex reconstruction and impact-parameter resolution. Increased pile-up and higher trigger rates motivated substantial advances in reconstruction software and trigger strategies.

---

\subsubsection{Run 3 (2022--2025): }

Run~3 operates at a center-of-mass energy of $\sqrt{s}=13.6~\mathrm{TeV}$, the highest achieved at the LHC to date. \cite{LHCRun3} Although the energy increase relative to Run~2 is modest, Run~3 is characterized by higher luminosity and more challenging pile-up conditions.

This run serves as a critical transition period toward the High-Luminosity LHC (HL-LHC). Upgraded trigger architectures, real-time reconstruction techniques, and improved tracking algorithms are validated under realistic conditions, preparing for Run~4. \cite{ATLASRun3Overview}

---

\subsubsection{Long Shutdown 3 and Run 4 (HL-LHC Era)}

Long Shutdown~3 (LS3) will enable the transformation of the LHC into the High-Luminosity LHC, with the goal of delivering an integrated luminosity of approximately $3$--$4~\mathrm{ab}^{-1}$. \cite{HLLHCDesign} ATLAS will undergo major Phase-II upgrades, including the complete replacement of the Inner Detector with the all-silicon Inner Tracker (ITk). \cite{ATLASITkTDR}

Run~4 will define the HL-LHC era, characterized by unprecedented pile-up levels and data volumes. These conditions require fundamentally new approaches to tracking, triggering, and data processing, including increased parallelization and hardware acceleration. Physics goals include percent-level measurements of the Higgs coupling, rare Standard Model processes, and enhanced sensitivity to new physics phenomena.

---

\subsection{The ATLAS Experimental Complex}

The ATLAS experiment is one of the four major detectors located at interaction points along the Large Hadron Collider (LHC) ring at CERN as in figure ~\ref{fig:cern}. It is designed as a general-purpose detector capable of recording the products of high-energy proton--proton collisions over a wide range of final states. The detector surrounds the interaction point and provides sufficient angular coverage for particle reconstruction.

The overall experimental setup integrates the detector with the LHC accelerator infrastructure, allowing proton beams circulating in opposite directions to collide at the center of ATLAS. The detector layout is optimized to measure charged-particle trajectories, electromagnetic and hadronic energy deposits, and muon momenta, enabling precise reconstruction of collision events. This integrated design supports stable detector operation during extended data-taking periods and accommodates progressively higher collision rates across successive LHC runs.

\begin{figure}[htb]
    \centering
    \includegraphics[width=0.8\textwidth]{figures/CCC-v2019-final-white.png}
    \caption{Schematic of CERN accelerator complex \cite{Mobs2019CERN}}
    \label{fig:cern}
\end{figure}

---

\subsubsection{Luminosity}

The physics reach of the ATLAS experiment is strongly influenced by the LHC's delivered luminosity. Luminosity characterizes the rate at which proton--proton collisions occur and determines the expected number of events for a given process with a known cross section. It is defined as
\begin{equation}
\mathcal{L} = \frac{N}{\sigma},
\end{equation}
where $N$ is the number of observed events and $\sigma$ is the corresponding production cross section.

The instantaneous luminosity depends on accelerator parameters such as the number of protons per bunch, the number of colliding bunches, and the transverse beam profiles at the interaction point. The total dataset collected by ATLAS is described by the integrated luminosity, obtained by integrating the instantaneous luminosity over time. Integrated luminosity is typically expressed in inverse femtobarns ($\mathrm{fb}^{-1}$) and represents the cumulative data available for physics analyses.

As the LHC has progressed through successive operational periods, improvements in accelerator performance have resulted in significant increases in delivered luminosity. While higher luminosity enhances sensitivity to rare processes, it also increases detector occupancy and makes reconstruction more challenging, leading to physics analysis.

\begin{figure}[htb]
    \centering
    \includegraphics[width=0.8\textwidth]{figures/intlumivstimeRun3.png}
    \caption{Cumulative integrated luminosity recorded by ATLAS during Run 3 \cite{ATLASRun3Luminosity}}
    \label{fig:lumi}
\end{figure}

---

\subsubsection{Pile-up}

Pile-up refers to multiple proton-proton interactions occurring within a single bunch crossing. It is commonly quantified by the parameter $\mu$, which represents the average number of inelastic interactions per bunch crossing. Higher luminosity operation of the LHC directly leads to increased pile-up.

Pile-up interactions contribute additional charged particles and energy deposits that are unrelated to the primary hard-scatter process and to the secondary vertex from b-quark decays. These additional interactions can affect track reconstruction, vertex identification, jet energy measurements, and missing transverse momentum reconstruction. As a result, pile-up represents a major experimental challenge, particularly in later LHC runs where luminosity is substantially higher.

To mitigate the impact of pile-up, ATLAS employs a combination of detector design features and reconstruction techniques, including precise vertex reconstruction and pile-up-aware object definitions. Accurate modeling of pile-up in Monte Carlo simulations is also essential, requiring simulated events to be reweighted to reproduce the pile-up conditions observed in data.

---

\subsection{The ATLAS Detector}
\begin{figure}[htb]
    \centering
    \includegraphics[width=0.8\textwidth]{figures/ATLAS Detector Schematic black particles.png}
    \caption{ATLAS detector slice with labeled subdetectors and particle visualizations \cite{Mehlhase:2770815}}
    \label{fig:detector}
\end{figure}

\subsubsection{Inner Detector}

The Inner Detector is the innermost tracking system in ATLAS and reconstructs the trajectories of charged particles produced in proton--proton collisions, which consists of pixel and strip detectors and Transition Radiation Tracker (TRT, and will be removed in Run 4). It operates in a uniform solenoidal magnetic field, enabling precise measurements of charged-particle momenta via curvature. The Inner Detector provides accurate measurements of track parameters, including transverse momentum, impact parameters, and vertex positions.

By combining information from multiple tracking layers, the Inner Detector achieves high efficiency for charged-particle reconstruction over a wide pseudorapidity range. Precise vertex reconstruction enables the identification of primary interaction vertices and the separation of tracks originating from pile-up interactions. The Inner Detector plays a central role in object identification, including electrons, charged hadrons, and tracks associated with jets.

---

\subsubsection{Calorimeter System}

The calorimeter system surrounds the Inner Detector and is designed to measure particle energies through interactions with dense absorber materials. It is divided into electromagnetic and hadronic calorimeters, each optimized for different particle types. The electromagnetic calorimeter provides high-resolution energy measurements for electrons and photons, while the hadronic calorimeter measures the energies of hadrons and jets.

The calorimeter system provides nearly full angular coverage and is essential for jet reconstruction, electron and photon identification, and missing transverse momentum determination. In addition to energy measurements, calorimeter information is used for particle identification and plays a key role in the trigger system.

---

\subsubsection{Muon Spectrometer}

The Muon Spectrometer forms the outermost detector system of ATLAS and is dedicated to the identification and momentum measurement of muons. Muons traverse the inner detector and calorimeters with minimal energy loss, allowing them to be measured in a large volume surrounding the calorimeter system.

The Muon Spectrometer employs large magnetic fields and multiple detector technologies to independently reconstruct muon trajectories. Momentum measurements from the Muon Spectrometer are combined with those from the Inner Detector to achieve improved resolution across a broad momentum range. The system provides robust muon identification and triggering capabilities, which are essential for many physics analyses.

---

\subsection{Trigger and Data Acquisition System}

The high collision rate delivered by the LHC far exceeds the rate at which events can be recorded for offline analysis. To address this challenge, ATLAS employs a trigger and data acquisition system that selects potentially interesting events in real time while rejecting the majority of collisions.

The trigger system reduces the event rate through a multi-level selection process based on information from the detector subsystems. Trigger decisions are designed to maintain high efficiency in the physics processes of interest while operating within strict latency and bandwidth constraints. The data acquisition system then records the selected events for further processing and offline reconstruction. Together, the trigger and data acquisition systems enable ATLAS to efficiently exploit the physics potential of the LHC.

\begin{figure}[htb]
    \centering
    \includegraphics[width=0.8\textwidth]{figures/tdaq.pdf}
    \caption{ATLAS Trigger and Data Acquisition System in Run 3 \cite{ATLASTriggerRun3}}
    \label{fig:Trigger}
\end{figure}

\section{ATLAS Run 4 Upgrade}

In Run 4, the Large Hadron Collider will enter the High-Luminosity LHC (HL-LHC) phase, which is designed to significantly extend the physics reach of the ATLAS experiment. The HL-LHC aims to deliver an integrated luminosity of approximately $3$--$4~\mathrm{ab}^{-1}$ over its operational lifetime, corresponding to more than an order-of-magnitude increase relative to the datasets collected in earlier runs. This increase will enable precise measurements of rare Standard Model processes and enhance sensitivity to physics beyond the Standard Model.

The higher luminosity of the HL-LHC will result in substantially more challenging experimental conditions. The instantaneous luminosity increases to an average pile-up level of up to $\langle\mu\rangle \sim 200$ interactions per bunch crossing. Such conditions will significantly increase detector occupancies, radiation exposure, and the complexity of event reconstruction. Maintaining the performance of tracking, vertexing, and triggering under these conditions requires major upgrades to the ATLAS detector systems.

A key component of the Phase-II upgrade program is the complete replacement of the current Inner Detector with a new all-silicon Inner Tracker (ITk). The ITk is designed to provide increased granularity, improved radiation tolerance, and extended pseudorapidity coverage compared to the existing detector. These improvements are essential for preserving tracking efficiency, momentum resolution, and vertex reconstruction performance in the high-pile-up environment of the HL-LHC.

In parallel, the ATLAS trigger and data acquisition system will be upgraded to cope with the increased event rates and data volumes expected during HL-LHC operation. The upgraded trigger system is designed to support higher output rates and more sophisticated real-time reconstruction algorithms, enabling efficient selection of physics events while operating within strict latency constraints.

The ATLAS Phase-II upgrade program is therefore essential for sustaining detector performance and physics sensitivity throughout the HL-LHC era. The design choices and technological developments implemented during this upgrade phase directly address the challenges posed by high luminosity and pile-up. \cite{HLLHCTDR} \cite{ATLASPhaseIITDR}


%--- Chapter 2 ----------------------------------------------------------------%
\chapter{ATLAS Event Filter (EF) Tracking: Strip Clustering}
\section{Background and motivation}
\subsection{Previous work and motivation}
\subsection{Overview of GPU-based Pipelines}
\subsection{\testtt{Traccc} and Athena framework}
\subsection{Strip clustering algorithm}

\section{Methodology}
\subsection{Input data}
\subsection{Integration with the \texttt{Traccc} framework}

\section{Results and Discussion}
\subsection{Cluster comparison with CPU}
\subsection{Efficiency}
\subsection{Summary of EF Tracking validation}

%%--- Chapter 3 ---------------------------------------------------------------%
\chapter{ATLAS: Signal Optimization of A Search for Higgs Decays to Dark Photons using ATLAS Run 3 Data}
\section{Introduction}

\subsection{Motivation for Hidden Sectors}

Despite the success of the Standard Model (SM) in describing particle interactions up to the electroweak scale, several observations indicate the existence of physics beyond the Standard Model. Astrophysical and cosmological measurements provide overwhelming evidence for dark matter, which constitutes approximately 27\% of the energy density of the Universe, yet does not participate in electromagnetic, strong, or weak interactions \cite{Planck2018}. This motivates the possibility that dark matter resides in a hidden sector containing new particles and forces that are weakly coupled to the SM.

A minimal and well-motivated extension of the SM introduces an additional Abelian gauge symmetry, $U(1)_D$, associated with a new gauge boson commonly referred to as the \emph{dark photon}. Such hidden-sector gauge fields naturally arise in many ultraviolet-complete theories, including string-inspired models and grand unified frameworks. \cite{Holdom1986,ArkaniHamed2009}

---

\subsection{Kinetic Mixing and the Dark Photon}

The dark photon, denoted $A'_\mu$, is associated with a new $U(1)_D$ gauge symmetry. The most general renormalizable Lagrangian describing the interaction between the SM hypercharge gauge field $B_\mu$ and the dark photon includes a kinetic mixing term \cite{Holdom1986}:
\begin{equation}
\mathcal{L} \supset -\frac{1}{4} F_{\mu\nu} F^{\mu\nu}
- \frac{1}{4} F'_{\mu\nu} F'^{\mu\nu}
- \frac{\epsilon}{2} F_{\mu\nu} F'^{\mu\nu},
\end{equation}
where $F_{\mu\nu}$ and $F'_{\mu\nu}$ are the field strength tensors of the SM electromagnetic field and the dark photon field, respectively, and $\epsilon$ is the kinetic mixing parameter.

This interaction is gauge invariant and can be generated radiatively through loops of heavy particles charged under both $U(1)_Y$ and $U(1)_D$. Natural values of $\epsilon$ span a wide range, typically $10^{-12} \lesssim \epsilon \lesssim 10^{-2}$, depending on the ultraviolet completion of the theory \cite{Essig2013}.

After diagonalizing the kinetic terms and performing electroweak symmetry breaking, the dark photon acquires an effective coupling to the electromagnetic current:
\begin{equation}
\mathcal{L}_{\text{int}} = \epsilon e A'_\mu J^\mu_{\text{EM}},
\end{equation}
allowing the dark photon to interact weakly with electrically charged SM particles.

---

\subsection{Massless and Massive Dark Photons}

The phenomenology of the dark photon depends critically on whether it is massless or massive.

\subsubsection{Massless Dark Photon}

If the $U(1)_D$ symmetry remains unbroken, the dark photon is massless. In this case, the kinetic mixing can be absorbed into a redefinition of the electric charge for particles charged under both gauge groups. As a result, a massless dark photon does not lead to observable effects in purely SM processes and is tightly constrained by charge quantization and precision measurements. \cite{Fabbrichesi2020}


\subsubsection{Massive Dark Photon}

If the $U(1)_D$ symmetry is broken, the dark photon acquires a mass $m_{A'}$ through either the Higgs mechanism or the Stueckelberg mechanism. In this case, the kinetic mixing term cannot be rotated away, and the dark photon becomes a physically observable particle. \cite{Holdom1986}

The effective Lagrangian for a massive dark photon is given by:
\begin{equation}
\mathcal{L} \supset \frac{1}{2} m_{A'}^2 A'_\mu A'^\mu + \epsilon e A'_\mu J^\mu_{\text{EM}}.
\end{equation}

The massive dark photon can decay into SM fermion pairs if kinematically allowed, with a decay width proportional to $\epsilon^2$.

\subsection{Dark Photons at the LHC}

At the Large Hadron Collider (LHC), dark photons can be produced through several mechanisms, including exotic Higgs decays, Drell--Yan processes, and meson decays. This thesis focuses on scenarios in which the dark photon is produced in association with a photon through Higgs boson decay from gluon gluon fusion
\begin{figure}[htb]
    \centering
    \includegraphics[width=0.6\textwidth]{figures/Feynman_ggF.png}
    \caption{Feynman diagram of gluon-gluon fusion Higgs boson, following decay into a SM photon and a dark photon ($gg \rightarrow H \rightarrow \gamma \gamma_D$).}
\label{fig:ggh_f}

\end{figure}
where $\gamma_D$ denotes a dark photon that decays invisibly or escapes detection.

Such processes arise naturally in models where the Higgs boson couples to both the SM photon and the dark photon via higher-dimensional operators or loop-induced interactions. \cite{Curtin2015} The experimental signature is characterized by a high-energy photon recoiling against missing transverse momentum.

Using the ATLAS detectors, we target events with a SM photon, missing transverse energy, and jets from initial gluon fusion. This thesis presents a detailed study of the sensitivity to the dark photon signal in the Run 3 environment of the LHC. 

\section{Data and MC samples}
\subsection{Data samples}

The data used in this analysis was collected in 2023 and 2024.
The used Good Run List are 
\begin{itemize}
    \item \texttt{data23\_13p6TeV.periodAllYear\_DetStatus-v133-pro31-11\_MERGED\_PHYS\_StandardGRL\_All\_Good\_25ns} for 2023 (25.204 \fbinv)
    \item \texttt{data24\_13p6TeV.periodsEtoO\_DetStatus-v130-pro36-08\_MERGED\_PHYS\_StandardGRL\_All\_Good\_25ns} for 2024 (109.40 \fbinv)
\end{itemize}

for a total integrated luminosity of 134.6 \fbinv, with an uncertainty of 2\%~\cite{lumi}. 

\subsection{MC samples}

The MC samples were processed with p-tag p6697.

The signal samples were showered from common Higgs LHE samples generated using Powheg. The decay process used in the shower was implemented with Pythia. The relevant JIRA tickets can be found at: \href{https://its.cern.ch/jira/browse/ATLMCPROD-10773}{ATLMCPROD-10733}, \href{https://its.cern.ch/jira/browse/ATLMCPROD-10819}{ATLMCPROD-10819}, \href{https://its.cern.ch/jira/browse/ATLMCPROD-10916}{ATLMCPROD-10916} and the sample validation at: \href{https://twiki.cern.ch/twiki/bin/view/AtlasProtected/MC23HLRSHyydValidation}{MC23HLRSHyydValidation}. The signal sample paths are shown in table~\ref{tab:signal-samples}. 

\begin{table}[ht]
\centering
\scriptsize
\caption{List of signal MCd samples used in this analysis.}
\label{tab:signal-samples}
\begin{tabular}{l}
\toprule
mc23\_13p6TeV.602403.PhPy8EG\_VBF\_H125yyd.deriv.DAOD\_PHYS.e8523\_e8528\_s4159\_s4114\_r15224\_r15225\_p6697\\
mc23\_13p6TeV.602913.PhPy8EG\_qqZH\_H125yyd.deriv.DAOD\_PHYS.e8523\_e8528\_s4159\_s4114\_r15224\_r15225\_p6697\\
mc23\_13p6TeV.602914.PhPy8EG\_WpH\_H125yyd.deriv.DAOD\_PHYS.e8523\_e8528\_s4159\_s4114\_r15224\_r15225\_p6697\\
mc23\_13p6TeV.602915.PhPy8EG\_WmH\_H125yyd.deriv.DAOD\_PHYS.e8523\_e8528\_s4159\_s4114\_r15224\_r15225\_p6697\\
mc23\_13p6TeV.602402.PhPy8EG\_ggH\_H125yyd.deriv.DAOD\_PHYS.e8537\_e8528\_s4159\_s4114\_r15224\_r15225\_p6320\\
\bottomrule
\end{tabular}
\end{table}

The primary backgrounds for the analysis include processes with a true photon in the final states and genuine \ETmiss. These samples include: $Z\gamma$ and $W\gamma$, where the vector boson can decay to a lepton that is not reconstructed or to neutrinos.
Additional background processes originate from a fake photon due to misidentified jets or electrons, and/or fake $\ETmiss$. 
The background due to jets or electrons misidentified as photons is mainly given by $Zjets$, $Wjets$, di-jets and $\gamma jets$ with fragmentation photon. The $Zjets$, di-jets and $\gamma(frag) jets$ events are also characterized by fake $\ETmiss$ due to mismeaured or not reconstructed jets. Finally, $\gamma jets$ with direct photon ($\gamma (direct)jets$) enter the analysis regions exclusively due to the presence of fake $\ETmiss$. \\
The list of MC samples used in this analysis is shown in table~\ref{tab:gammajet-samples} for $\gamma(direct) jets$ and table~\ref{tab:Vy-samples} for $W\gamma/Z\gamma$, table~\ref{tab:V-samples} for $W/Z$. As detailed later, in section \ref{sec:bkg}, MC simulations are eventually employed only for true photon backgrounds (including the $\gamma(direct) jets$), while the contribution from events with fake photons are estimated through data-driven methods. The $W/Z$ MC samples are used as part of the \jfakey method and therefore listed in the following tables, while $\gamma(direct) jets$ and di-jets have been used exclusively for preliminary studies and initial optimization of the analysis selection and are not reported here.  

\begin{table}[ht]
\centering
\scriptsize
\caption{List of $\gamma jets$ MC23d and MC23e samples used in this analysis.}
\label{tab:gammajet-samples}
\begin{tabular}{l}
\toprule
MC23d \\
\midrule
mc23\_13p6TeV.801663.Py8\_gammajet\_direct\_DP8\_17\_FullSW.deriv.DAOD\_PHYS.e8514\_e8528\_s4159\_s4114\_r15224\_r15225\_p6697\\
mc23\_13p6TeV.801664.Py8\_gammajet\_direct\_DP17\_35\_FullSW.deriv.DAOD\_PHYS.e8514\_e8528\_s4159\_s4114\_r15224\_r15225\_p6697\\
mc23\_13p6TeV.801665.Py8\_gammajet\_direct\_DP35\_50\_FullSW.deriv.DAOD\_PHYS.e8514\_e8528\_s4159\_s4114\_r15224\_r15225\_p6697\\
mc23\_13p6TeV.801666.Py8\_gammajet\_direct\_DP50\_70\_FullSW.deriv.DAOD\_PHYS.e8514\_e8528\_s4159\_s4114\_r15224\_r15225\_p6697\\
mc23\_13p6TeV.801667.Py8\_gammajet\_direct\_DP70\_140\_FullSW.deriv.DAOD\_PHYS.e8514\_e8528\_s4159\_s4114\_r15224\_r15225\_p6697\\
mc23\_13p6TeV.801668.Py8\_gammajet\_direct\_DP140\_280\_FullSW.deriv.DAOD\_PHYS.e8514\_e8528\_s4159\_s4114\_r15224\_r15225\_p6697\\
mc23\_13p6TeV.801669.Py8\_gammajet\_direct\_DP280\_500\_FullSW.deriv.DAOD\_PHYS.e8514\_e8528\_s4159\_s4114\_r15224\_r15225\_p6697\\
mc23\_13p6TeV.801670.Py8\_gammajet\_direct\_DP500\_800\_FullSW.deriv.DAOD\_PHYS.e8514\_e8528\_s4159\_s4114\_r15224\_r15225\_p6697\\
mc23\_13p6TeV.801671.Py8\_gammajet\_direct\_DP800\_1000\_FullSW.deriv.DAOD\_PHYS.e8514\_e8528\_s4159\_s4114\_r15224\_r15225\_p6697\\
mc23\_13p6TeV.801672.Py8\_gammajet\_direct\_DP1000\_1500\_FullSW.deriv.DAOD\_PHYS.e8514\_e8528\_s4159\_s4114\_r15224\_r15225\_p6697\\
mc23\_13p6TeV.801673.Py8\_gammajet\_direct\_DP1500\_2000\_FullSW.deriv.DAOD\_PHYS.e8514\_e8528\_s4159\_s4114\_r15224\_r15225\_p6697\\
mc23\_13p6TeV.801674.Py8\_gammajet\_direct\_DP2000\_2500\_FullSW.deriv.DAOD\_PHYS.e8514\_e8528\_s4159\_s4114\_r15224\_r15225\_p6697\\
mc23\_13p6TeV.801675.Py8\_gammajet\_direct\_DP2500\_3000\_FullSW.deriv.DAOD\_PHYS.e8514\_e8528\_s4159\_s4114\_r15224\_r15225\_p6697\\
mc23\_13p6TeV.801676.Py8\_gammajet\_direct\_DP3000\_inf\_FullSW.deriv.DAOD\_PHYS.e8514\_e8528\_s4159\_s4114\_r15224\_r15225\_p6697\\
\midrule
MC23e \\
\midrule
mc23\_13p6TeV.801663.Py8\_gammajet\_direct\_DP8\_17\_FullSW.deriv.DAOD\_PHYS.e8514\_e8528\_s4369\_s4370\_r16083\_r15970\_p6697\\
mc23\_13p6TeV.801664.Py8\_gammajet\_direct\_DP17\_35\_FullSW.deriv.DAOD\_PHYS.e8514\_e8528\_s4369\_s4370\_r16083\_r15970\_p6697\\
mc23\_13p6TeV.801665.Py8\_gammajet\_direct\_DP35\_50\_FullSW.deriv.DAOD\_PHYS.e8514\_e8528\_s4369\_s4370\_r16083\_r15970\_p6697\\
mc23\_13p6TeV.801666.Py8\_gammajet\_direct\_DP50\_70\_FullSW.deriv.DAOD\_PHYS.e8514\_e8528\_s4369\_s4370\_r16083\_r15970\_p6697\\
mc23\_13p6TeV.801667.Py8\_gammajet\_direct\_DP70\_140\_FullSW.deriv.DAOD\_PHYS.e8514\_e8528\_s4369\_s4370\_r16083\_r15970\_p6697\\
mc23\_13p6TeV.801668.Py8\_gammajet\_direct\_DP140\_280\_FullSW.deriv.DAOD\_PHYS.e8514\_e8528\_s4369\_s4370\_r16083\_r15970\_p6697\\
mc23\_13p6TeV.801669.Py8\_gammajet\_direct\_DP280\_500\_FullSW.deriv.DAOD\_PHYS.e8514\_e8528\_s4369\_s4370\_r16083\_r15970\_p6697\\
mc23\_13p6TeV.801670.Py8\_gammajet\_direct\_DP500\_800\_FullSW.deriv.DAOD\_PHYS.e8514\_e8528\_s4369\_s4370\_r16083\_r15970\_p6697\\
mc23\_13p6TeV.801671.Py8\_gammajet\_direct\_DP800\_1000\_FullSW.deriv.DAOD\_PHYS.e8514\_e8528\_s4369\_s4370\_r16083\_r15970\_p6697\\
mc23\_13p6TeV.801672.Py8\_gammajet\_direct\_DP1000\_1500\_FullSW.deriv.DAOD\_PHYS.e8514\_e8528\_s4369\_s4370\_r16083\_r15970\_p6697\\
mc23\_13p6TeV.801673.Py8\_gammajet\_direct\_DP1500\_2000\_FullSW.deriv.DAOD\_PHYS.e8514\_e8528\_s4369\_s4370\_r16083\_r15970\_p6697\\
mc23\_13p6TeV.801674.Py8\_gammajet\_direct\_DP2000\_2500\_FullSW.deriv.DAOD\_PHYS.e8514\_e8528\_s4369\_s4370\_r16083\_r15970\_p6697\\
mc23\_13p6TeV.801675.Py8\_gammajet\_direct\_DP2500\_3000\_FullSW.deriv.DAOD\_PHYS.e8514\_e8528\_s4369\_s4370\_r16083\_r15970\_p6697\\
mc23\_13p6TeV.801676.Py8\_gammajet\_direct\_DP3000\_inf\_FullSW.deriv.DAOD\_PHYS.e8514\_e8528\_s4369\_s4370\_r16083\_r15970\_p6697\\ 
\bottomrule
\end{tabular}
\end{table}


\begin{table}[ht]
\centering
\scriptsize
\caption{List of $W\gamma/Z\gamma$ MC23d and MC23e samples used in this analysis}
\label{tab:Vy-samples}
\begin{tabular}{l}
\toprule
MC23d \\
\midrule
mc23\_13p6TeV.700770.Sh\_2214\_eegamma.deriv.DAOD\_PHYS.e8514\_e8528\_s4159\_s4114\_r15224\_r15225\_p6697\\
mc23\_13p6TeV.700771.Sh\_2214\_mumugamma.deriv.DAOD\_PHYS.e8514\_e8528\_s4159\_s4114\_r15224\_r15225\_p6697\\
mc23\_13p6TeV.700772.Sh\_2214\_tautaugamma.deriv.DAOD\_PHYS.e8514\_e8528\_s4159\_s4114\_r15224\_r15225\_p6697\\
mc23\_13p6TeV.700773.Sh\_2214\_enugamma.deriv.DAOD\_PHYS.e8514\_e8528\_s4159\_s4114\_r15224\_r15225\_p6697\\
mc23\_13p6TeV.700774.Sh\_2214\_munugamma.deriv.DAOD\_PHYS.e8514\_e8528\_s4159\_s4114\_r15224\_r15225\_p6697\\
mc23\_13p6TeV.700775.Sh\_2214\_taunugamma.deriv.DAOD\_PHYS.e8514\_e8528\_s4159\_s4114\_r15224\_r15225\_p6697\\
mc23\_13p6TeV.700776.Sh\_2214\_nunugamma.deriv.DAOD\_PHYS.e8514\_e8528\_s4159\_s4114\_r15224\_r15225\_p6697\\
\midrule
MC23e \\
\midrule
mc23\_13p6TeV.700770.Sh\_2214\_eegamma.deriv.DAOD\_PHYS.e8514\_e8528\_s4369\_s4370\_r16083\_r15970\_p6697\\
mc23\_13p6TeV.700771.Sh\_2214\_mumugamma.deriv.DAOD\_PHYS.e8514\_e8528\_s4369\_s4370\_r16083\_r15970\_p6697\\
mc23\_13p6TeV.700772.Sh\_2214\_tautaugamma.deriv.DAOD\_PHYS.e8514\_e8528\_s4369\_s4370\_r16083\_r15970\_p6697\\
mc23\_13p6TeV.700773.Sh\_2214\_enugamma.deriv.DAOD\_PHYS.e8514\_e8528\_s4369\_s4370\_r16083\_r15970\_p6697\\
mc23\_13p6TeV.700774.Sh\_2214\_munugamma.deriv.DAOD\_PHYS.e8514\_e8528\_s4369\_s4370\_r16083\_r15970\_p6697\\
mc23\_13p6TeV.700775.Sh\_2214\_taunugamma.deriv.DAOD\_PHYS.e8514\_e8528\_s4369\_s4370\_r16083\_r15970\_p6697\\
mc23\_13p6TeV.700776.Sh\_2214\_nunugamma.deriv.DAOD\_PHYS.e8514\_e8528\_s4369\_s4370\_r16083\_r15970\_p6697\\
\bottomrule
\end{tabular}
\end{table}

\begin{table}[ht]
\centering
\scriptsize
\caption{List of $W/Z$ MC23d and MC23e samples used in this analysis.}
\label{tab:V-samples}
\begin{tabular}{l}
\toprule
MC23d \\
\midrule
mc23\_13p6TeV.700777.Sh\_2214\_Wenu\_maxHTpTV2\_BFilter.deriv.DAOD\_PHYS.e8514\_e8528\_s4159\_s4114\_r15224\_r15225\_p6697 \\
mc23\_13p6TeV.700778.Sh\_2214\_Wenu\_maxHTpTV2\_CFilterBVeto.deriv.DAOD\_PHYS.e8514\_e8528\_s4159\_s4114\_r15224\_r15225\_p6697 \\
mc23\_13p6TeV.700779.Sh\_2214\_Wenu\_maxHTpTV2\_CVetoBVeto.deriv.DAOD\_PHYS.e8514\_e8528\_s4159\_s4114\_r15530\_r15514\_p6697 \\
mc23\_13p6TeV.700780.Sh\_2214\_Wmunu\_maxHTpTV2\_BFilter.deriv.DAOD\_PHYS.e8514\_e8528\_s4159\_s4114\_r15224\_r15225\_p6697 \\
mc23\_13p6TeV.700781.Sh\_2214\_Wmunu\_maxHTpTV2\_CFilterBVeto.deriv.DAOD\_PHYS.e8514\_e8528\_s4159\_s4114\_r15224\_r15225\_p6697 \\
mc23\_13p6TeV.700782.Sh\_2214\_Wmunu\_maxHTpTV2\_CVetoBVeto.deriv.DAOD\_PHYS.e8514\_e8528\_s4159\_s4114\_r15530\_r15514\_p6697 \\
mc23\_13p6TeV.700783.Sh\_2214\_Wtaunu\_maxHTpTV2\_BFilter.deriv.DAOD\_PHYS.e8514\_e8528\_s4159\_s4114\_r15224\_r15225\_p6697 \\
mc23\_13p6TeV.700784.Sh\_2214\_Wtaunu\_maxHTpTV2\_CFilterBVeto.deriv.DAOD\_PHYS.e8514\_e8528\_s4159\_s4114\_r15224\_r15225\_p6697 \\
mc23\_13p6TeV.700785.Sh\_2214\_Wtaunu\_maxHTpTV2\_CVetoBVeto.deriv.DAOD\_PHYS.e8514\_e8528\_s4159\_s4114\_r15530\_r15514\_p6697 \\
mc23\_13p6TeV.700786.Sh\_2214\_Zee\_maxHTpTV2\_BFilter.deriv.DAOD\_PHYS.e8514\_e8528\_s4159\_s4114\_r15224\_r15225\_p6697 \\
mc23\_13p6TeV.700787.Sh\_2214\_Zee\_maxHTpTV2\_CFilterBVeto.deriv.DAOD\_PHYS.e8514\_e8528\_s4159\_s4114\_r15224\_r15225\_p6697 \\
mc23\_13p6TeV.700788.Sh\_2214\_Zee\_maxHTpTV2\_CVetoBVeto.deriv.DAOD\_PHYS.e8514\_e8528\_s4159\_s4114\_r15530\_r15514\_p6697 \\
mc23\_13p6TeV.700789.Sh\_2214\_Zmumu\_maxHTpTV2\_BFilter.deriv.DAOD\_PHYS.e8514\_e8528\_s4159\_s4114\_r15224\_r15225\_p6697 \\
mc23\_13p6TeV.700790.Sh\_2214\_Zmumu\_maxHTpTV2\_CFilterBVeto.deriv.DAOD\_PHYS.e8514\_e8528\_s4159\_s4114\_r15224\_r15225\_p6697 \\
mc23\_13p6TeV.700791.Sh\_2214\_Zmumu\_maxHTpTV2\_CVetoBVeto.deriv.DAOD\_PHYS.e8514\_e8528\_s4159\_s4114\_r15530\_r15514\_p6697 \\
mc23\_13p6TeV.700792.Sh\_2214\_Ztautau\_maxHTpTV2\_BFilter.deriv.DAOD\_PHYS.e8514\_e8528\_s4159\_s4114\_r15224\_r15225\_p6697 \\
mc23\_13p6TeV.700793.Sh\_2214\_Ztautau\_maxHTpTV2\_CFilterBVeto.deriv.DAOD\_PHYS.e8514\_e8528\_s4159\_s4114\_r15224\_r15225\_p6697 \\
mc23\_13p6TeV.700794.Sh\_2214\_Ztautau\_maxHTpTV2\_CVetoBVeto.deriv.DAOD\_PHYS.e8514\_e8528\_s4159\_s4114\_r15530\_r15514\_p6697 \\
mc23\_13p6TeV.700795.Sh\_2214\_Znunu\_pTV2\_BFilter.deriv.DAOD\_PHYS.e8514\_e8528\_s4159\_s4114\_r15224\_r15225\_p6697 \\
mc23\_13p6TeV.700796.Sh\_2214\_Znunu\_pTV2\_CFilterBVeto.deriv.DAOD\_PHYS.e8514\_e8528\_s4159\_s4114\_r15224\_r15225\_p6697 \\
mc23\_13p6TeV.700797.Sh\_2214\_Znunu\_pTV2\_CVetoBVeto.deriv.DAOD\_PHYS.e8514\_e8528\_s4159\_s4114\_r15530\_r15514\_p6697 \\
\midrule
MC23e \\
\midrule
mc23\_13p6TeV.700777.Sh\_2214\_Wenu\_maxHTpTV2\_BFilter.deriv.DAOD\_PHYS.e8514\_e8528\_s4369\_s4370\_r16083\_r15970\_p6697 \\
mc23\_13p6TeV.700778.Sh\_2214\_Wenu\_maxHTpTV2\_CFilterBVeto.deriv.DAOD\_PHYS.e8514\_e8528\_s4369\_s4370\_r16083\_r15970\_p6697 \\
mc23\_13p6TeV.700779.Sh\_2214\_Wenu\_maxHTpTV2\_CVetoBVeto.deriv.DAOD\_PHYS.e8514\_e8528\_s4159\_s4114\_r15530\_r15514\_p6697 \\
mc23\_13p6TeV.700780.Sh\_2214\_Wmunu\_maxHTpTV2\_BFilter.deriv.DAOD\_PHYS.e8514\_e8528\_s4369\_s4370\_r16083\_r15970\_p6697 \\
mc23\_13p6TeV.700781.Sh\_2214\_Wmunu\_maxHTpTV2\_CFilterBVeto.deriv.DAOD\_PHYS.e8514\_e8528\_s4369\_s4370\_r16083\_r15970\_p6697 \\
mc23\_13p6TeV.700782.Sh\_2214\_Wmunu\_maxHTpTV2\_CVetoBVeto.deriv.DAOD\_PHYS.e8514\_e8528\_s4159\_s4114\_r15530\_r15514\_p6697 \\
mc23\_13p6TeV.700783.Sh\_2214\_Wtaunu\_maxHTpTV2\_BFilter.deriv.DAOD\_PHYS.e8514\_e8528\_s4369\_s4370\_r16083\_r15970\_p6697 \\
mc23\_13p6TeV.700784.Sh\_2214\_Wtaunu\_maxHTpTV2\_CFilterBVeto.deriv.DAOD\_PHYS.e8514\_e8528\_s4369\_s4370\_r16083\_r15970\_p6697 \\
mc23\_13p6TeV.700785.Sh\_2214\_Wtaunu\_maxHTpTV2\_CVetoBVeto.deriv.DAOD\_PHYS.e8514\_e8528\_s4159\_s4114\_r15530\_r15514\_p6697 \\
mc23\_13p6TeV.700786.Sh\_2214\_Zee\_maxHTpTV2\_BFilter.deriv.DAOD\_PHYS.e8514\_e8528\_s4369\_s4370\_r16083\_r15970\_p6697 \\
mc23\_13p6TeV.700787.Sh\_2214\_Zee\_maxHTpTV2\_CFilterBVeto.deriv.DAOD\_PHYS.e8514\_e8528\_s4369\_s4370\_r16083\_r15970\_p6697 \\
mc23\_13p6TeV.700788.Sh\_2214\_Zee\_maxHTpTV2\_CVetoBVeto.deriv.DAOD\_PHYS.e8514\_e8528\_s4159\_s4114\_r15530\_r15514\_p6697 \\
mc23\_13p6TeV.700789.Sh\_2214\_Zmumu\_maxHTpTV2\_BFilter.deriv.DAOD\_PHYS.e8514\_e8528\_s4369\_s4370\_r16083\_r15970\_p6697 \\
mc23\_13p6TeV.700790.Sh\_2214\_Zmumu\_maxHTpTV2\_CFilterBVeto.deriv.DAOD\_PHYS.e8514\_e8528\_s4369\_s4370\_r16083\_r15970\_p6697 \\
mc23\_13p6TeV.700791.Sh\_2214\_Zmumu\_maxHTpTV2\_CVetoBVeto.deriv.DAOD\_PHYS.e8514\_e8528\_s4159\_s4114\_r15530\_r15514\_p6697 \\
mc23\_13p6TeV.700792.Sh\_2214\_Ztautau\_maxHTpTV2\_BFilter.deriv.DAOD\_PHYS.e8514\_e8528\_s4369\_s4370\_r16083\_r15970\_p6697 \\
mc23\_13p6TeV.700793.Sh\_2214\_Ztautau\_maxHTpTV2\_CFilterBVeto.deriv.DAOD\_PHYS.e8514\_e8528\_s4369\_s4370\_r16083\_r15970\_p6697 \\
mc23\_13p6TeV.700794.Sh\_2214\_Ztautau\_maxHTpTV2\_CVetoBVeto.deriv.DAOD\_PHYS.e8514\_e8528\_s4159\_s4114\_r15530\_r15514\_p6697 \\
mc23\_13p6TeV.700795.Sh\_2214\_Znunu\_pTV2\_BFilter.deriv.DAOD\_PHYS.e8514\_e8528\_s4369\_s4370\_r16083\_r15970\_p6697 \\
mc23\_13p6TeV.700796.Sh\_2214\_Znunu\_pTV2\_CFilterBVeto.deriv.DAOD\_PHYS.e8514\_e8528\_s4369\_s4370\_r16083\_r15970\_p6697 \\
mc23\_13p6TeV.700797.Sh\_2214\_Znunu\_pTV2\_CVetoBVeto.deriv.DAOD\_PHYS.e8514\_e8528\_s4159\_s4114\_r15530\_r15514\_p6697 \\
\bottomrule
\end{tabular}
\end{table}




\section{Physics object reconstruction and identification}
\subsection{Photons}
\subsection{Muons}
\subsection{Electrons}
\subsection{Jets}
\subsection{Overlap removal}
\subsection{Missing transverse momentum}

\section{Event selection}
\subsection{Trigger}
\subsection{Pre-selection}
\subsection{Signal region}

\section{Background modeling and estimation}
\subsection{Real photon backgrounds}
\subsection{Fake photon backgrounds}
\subsubsection{Jets faking photons}
\subsubsection{Electrons faking photons}

\section{Signal region optimization}
\subsection{N-1 iteration method}

The set of rectangular cuts resulting from the previous optimization is the following:

\noindent\textbf{Preselections:}
\begin{itemize}[topsep=2pt, itemsep=2pt, parsep=0pt]
    \item Analysis trigger: \texttt{HLT\_50ight\allowbreak
z40\_eLL\allowbreak
z70\_fopupfit8\allowbreak
\_mTAC\_LeEM26M}
    \item $N_{\gamma} = 1$, with $N_{\gamma}$ the number of selected photons;
    \item $N_{\text{leptons}} = 0$, for baseline leptons;
    \item $N_{\text{jet, central}} \leq 3$;
    \item $p_{T}^{\gamma} > 50~\text{GeV}$;
    \item $E_{T}^{\text{miss}} > 100~\text{GeV}$;
    \item $m_T(E_{T}^{\text{miss}}, \gamma) > 80~\text{GeV}$;
\end{itemize}
\noindent\textbf{Additional selections:}
\begin{itemize}[topsep=2pt, itemsep=2pt, parsep=0pt]
    \item $\mathrm{BDT}^{\text{vertex}}_{\text{score}} > 0.1$;
    \item $E_{T}^{\text{miss}}$ significance $> 6$;
    \item $\Delta E_{T}^{\text{miss}} > -10~\text{GeV}$;
    \item $\Delta \phi(E_{T}^{\text{miss}}, E_{T}^{\gamma}) > 1.25$;
    \item $\Delta \phi(E_{T}^{\text{miss}}, E_{T}^{\text{jet}}) < 0.75$;
    \item $\Delta \phi(j_1, j_2) < 2.5$;
    \item $|\eta_{\gamma}| < 1.75$;
\end{itemize}

After checking n-1 plots of each variable, we optimized the cuts to:
\noindent\textbf{Preselections:}
\begin{itemize}[topsep=2pt, itemsep=2pt, parsep=0pt]
    \item Analysis trigger: \texttt{HLT\_50ight\allowbreak
z40\_eLL\allowbreak
z70\_fopupfit8\allowbreak
\_mTAC\_LeEM26M}
    \item $N_{\gamma} = 1$, with $N_{\gamma}$ the number of selected photons;
    \item $N_{\text{leptons}} = 0$, for baseline leptons;
    \item $N_{\text{jet, central}} \leq 3$;
    \item $p_{T}^{\gamma} > 50~\text{GeV}$;
    \item $E_{T}^{\text{miss}} > 100~\text{GeV}$;
    \item $140~\text{GeV} > m_T(E_{T}^{\text{miss}}, \gamma) > 100~\text{GeV}$;
\end{itemize}
\noindent\textbf{Additional selections:}
\begin{itemize}[topsep=2pt, itemsep=2pt, parsep=0pt]
    \item $\mathrm{BDT}^{\text{vertex}}_{\text{score}} > 0.1$;
    \item $E_{T}^{\text{miss}}$ significance $> 7$;
    \item $\Delta E_{T}^{\text{miss}} > -25~\text{GeV}$;
    \item $\Delta \phi(E_{T}^{\text{miss}}, E_{T}^{\gamma}) > 1.25$;
    \item $\Delta \phi(E_{T}^{\text{miss}}, E_{T}^{\text{jet}}) < 0.75$;
    \item $\Delta \phi(j_1, j_2) < 2.5$;
    \item $|\eta_{\gamma}| < 1.75$;
    \item $balance > 0.9$;
\end{itemize}

With the improved optimized cuts, we increased the simple significance $ \frac{S}{\sqrt{B}} $ from 2.475 to 2.701. 

\begin{figure}[htbp]
  \centering

  % Row 1
  \begin{subfigure}[t]{0.32\textwidth}
    \centering\includegraphics[width=\linewidth]{figures/performance_plot/VertexBDTScore.png}
    \caption{$\mathrm{BDT}^{\text{vertex}}_{\text{score}}$}
    \label{fig:n1-vertexbdt}
  \end{subfigure}
  \begin{subfigure}[t]{0.32\textwidth}
    \centering\includegraphics[width=\linewidth]{figures/performance_plot/ph_eta.png}
    \caption{$|\eta_{\gamma}|$}
    \label{fig:n1-ph-eta}
  \end{subfigure}\hfill
  \begin{subfigure}[t]{0.32\textwidth}
    \centering\includegraphics[width=\linewidth]{figures/performance_plot/metsig.png}
    \caption{$E_{T}^{\text{miss}}$ significance}
    \label{fig:n1-metsig}
  \end{subfigure}

  \par\medskip

  % Row 2
  \begin{subfigure}[t]{0.32\textwidth}
    \centering\includegraphics[width=\linewidth]{figures/performance_plot/balance.png}
    \caption{Balance}
    \label{fig:n1-balance}
  \end{subfigure}
  \begin{subfigure}[t]{0.32\textwidth}
    \centering\includegraphics[width=\linewidth]{figures/performance_plot/dphi_met_phterm.png}
    \caption{$\Delta \phi(E_{T}^{\text{miss}}, E_{T}^{\gamma})$}
    \label{fig:n1-dphi-met-phterm}
  \end{subfigure}\hfill
  \begin{subfigure}[t]{0.32\textwidth}
    \centering\includegraphics[width=\linewidth]{figures/performance_plot/dphi_met_jetterm.png}
    \caption{$\Delta \phi(E_{T}^{\text{miss}}, E_{T}^{\text{jet}})$}
    \label{fig:n1-dphi-met-jetterm}
  \end{subfigure}

  \par\medskip

  % Row 3
  \begin{subfigure}[t]{0.32\textwidth}
    \centering\includegraphics[width=\linewidth]{figures/performance_plot/dphi_jj.png}
    \caption{$\Delta \phi(j_1, j_2)$}
    \label{fig:n1-dphi-jj}
  \end{subfigure}
  \begin{subfigure}[t]{0.32\textwidth}
    \centering\includegraphics[width=\linewidth]{figures/performance_plot/dmet.png}
    \caption{$\Delta E_{T}^{\text{miss}}$}
    \label{fig:n1-dmet}
  \end{subfigure}\hfill

  \par\medskip



  \caption{Additional Selection variables distributions after the optimized rectangular cuts for the ggF $H \to \gamma\gamma_D$ analysis.}
  \label{fig:n1-all}
\end{figure}


\begin{figure}[htbp]
  \centering

  % Row 1
  \begin{subfigure}[t]{0.32\textwidth}
    \centering\includegraphics[width=\linewidth]{figures/n-1_plot/VertexBDTScore.png}
    \caption{$\mathrm{BDT}^{\text{vertex}}_{\text{score}}$}
    \label{fig:n1-vertexbdt}
  \end{subfigure}
  \begin{subfigure}[t]{0.32\textwidth}
    \centering\includegraphics[width=\linewidth]{figures/n-1_plot/ph_eta.png}
    \caption{$|\eta_{\gamma}|$}
    \label{fig:n1-ph-eta}
  \end{subfigure}\hfill
  \begin{subfigure}[t]{0.32\textwidth}
    \centering\includegraphics[width=\linewidth]{figures/n-1_plot/metsig.png}
    \caption{$E_{T}^{\text{miss}}$ significance}
    \label{fig:n1-metsig}
  \end{subfigure}

  \par\medskip

  % Row 2
  \begin{subfigure}[t]{0.32\textwidth}
    \centering\includegraphics[width=\linewidth]{figures/n-1_plot/balance.png}
    \caption{Balance}
    \label{fig:n1-balance}
  \end{subfigure}
  \begin{subfigure}[t]{0.32\textwidth}
    \centering\includegraphics[width=\linewidth]{figures/n-1_plot/dphi_met_phterm.png}
    \caption{$\Delta \phi(E_{T}^{\text{miss}}, E_{T}^{\gamma})$}
    \label{fig:n1-dphi-met-phterm}
  \end{subfigure}\hfill
  \begin{subfigure}[t]{0.32\textwidth}
    \centering\includegraphics[width=\linewidth]{figures/n-1_plot/dphi_met_jetterm.png}
    \caption{$\Delta \phi(E_{T}^{\text{miss}}, E_{T}^{\text{jet}})$}
    \label{fig:n1-dphi-met-jetterm}
  \end{subfigure}

  \par\medskip

  % Row 3
  \begin{subfigure}[t]{0.32\textwidth}
    \centering\includegraphics[width=\linewidth]{figures/n-1_plot/dphi_jj.png}
    \caption{$\Delta \phi(j_1, j_2)$}
    \label{fig:n1-dphi-jj}
  \end{subfigure}
  \begin{subfigure}[t]{0.32\textwidth}
    \centering\includegraphics[width=\linewidth]{figures/n-1_plot/dmet.png}
    \caption{$\Delta E_{T}^{\text{miss}}$}
    \label{fig:n1-dmet}
  \end{subfigure}\hfill

  \par\medskip


  \caption{N$-1$ distributions for the ggF $H \to \gamma\gamma_D$ analysis.}
  \label{fig:n1-all}
\end{figure}
  


 
\subsection{Machine Learning approach}
\subsubsection{Boosted Decision Trees (BDT)}
\subsubsection{Neural Networks (NN)}

\section{Results and discussion}
\section{Conclusions and Outlook}

